\documentclass[12pt, fullpage,letterpaper]{article}

\usepackage[margin=1in]{geometry}
\usepackage{url}
\usepackage{amsmath}
\usepackage{amssymb}
\usepackage{xspace}
\usepackage{graphicx}

\newcommand{\semester}{Spring 2022}
\newcommand{\assignmentId}{0}
\newcommand{\releaseDate}{10 Jan, 2022}
\newcommand{\dueDate}{11:59pm, 21 Jan, 2022}

\newcommand{\bx}{{\bf x}}
\newcommand{\bw}{{\bf w}}

\title{CS 6190: Probabilistic Machine Learning \semester}
\author{Homework \assignmentId}
\date{Handed out: \releaseDate\\
  Due: \dueDate}

\begin{document}
\maketitle

% Math commands by Thomas Minka
\newcommand{\var}{{\rm var}}
\newcommand{\Tr}{^{\rm T}}
\newcommand{\vtrans}[2]{{#1}^{(#2)}}
\newcommand{\kron}{\otimes}
\newcommand{\schur}[2]{({#1} | {#2})}
\newcommand{\schurdet}[2]{\left| ({#1} | {#2}) \right|}
\newcommand{\had}{\circ}
\newcommand{\diag}{{\rm diag}}
\newcommand{\invdiag}{\diag^{-1}}
\newcommand{\rank}{{\rm rank}}
% careful: ``null'' is already a latex command
\newcommand{\nullsp}{{\rm null}}
\newcommand{\tr}{{\rm tr}}
\renewcommand{\vec}{{\rm vec}}
\newcommand{\vech}{{\rm vech}}
\renewcommand{\det}[1]{\left| #1 \right|}
\newcommand{\pdet}[1]{\left| #1 \right|_{+}}
\newcommand{\pinv}[1]{#1^{+}}
\newcommand{\erf}{{\rm erf}}
\newcommand{\hypergeom}[2]{{}_{#1}F_{#2}}

% boldface characters
\renewcommand{\a}{{\bf a}}
\renewcommand{\b}{{\bf b}}
\renewcommand{\c}{{\bf c}}
\renewcommand{\d}{{\rm d}}  % for derivatives
\newcommand{\e}{{\bf e}}
\newcommand{\f}{{\bf f}}
\newcommand{\g}{{\bf g}}
\newcommand{\h}{{\bf h}}
%\newcommand{\k}{{\bf k}}
% in Latex2e this must be renewcommand
\renewcommand{\k}{{\bf k}}
\newcommand{\m}{{\bf m}}
\newcommand{\mb}{{\bf m}}
\newcommand{\n}{{\bf n}}
\renewcommand{\o}{{\bf o}}
\newcommand{\p}{{\bf p}}
\newcommand{\q}{{\bf q}}
\renewcommand{\r}{{\bf r}}
\newcommand{\s}{{\bf s}}
\renewcommand{\t}{{\bf t}}
\renewcommand{\u}{{\bf u}}
\renewcommand{\v}{{\bf v}}
\newcommand{\w}{{\bf w}}
\newcommand{\x}{{\bf x}}
\newcommand{\y}{{\bf y}}
\newcommand{\z}{{\bf z}}
%s\newcommand{\l}{\boldsymbol{l}}
\newcommand{\A}{{\bf A}}
\newcommand{\B}{{\bf B}}
\newcommand{\C}{{\bf C}}
\newcommand{\D}{{\bf D}}
\newcommand{\E}{{\bf E}}
\newcommand{\F}{{\bf F}}
\newcommand{\G}{{\bf G}}
\renewcommand{\H}{{\bf H}}
\newcommand{\I}{{\bf I}}
\newcommand{\J}{{\bf J}}
\newcommand{\K}{{\bf K}}
\renewcommand{\L}{{\bf L}}
\newcommand{\M}{{\bf M}}
\newcommand{\N}{\mathcal{N}}  % for normal density
\newcommand{\Dcal}{\mathcal{D}}  % for normal density

%\newcommand{\N}{{\bf N}}
\renewcommand{\O}{{\bf O}}
\renewcommand{\P}{{\bf P}}
\newcommand{\Q}{{\bf Q}}
\newcommand{\R}{{\bf R}}
\renewcommand{\S}{{\bf S}}
\newcommand{\T}{{\bf T}}
\newcommand{\U}{{\bf U}}
\newcommand{\V}{{\bf V}}
\newcommand{\W}{{\bf W}}
\newcommand{\X}{{\bf X}}
\newcommand{\Y}{{\bf Y}}
\newcommand{\Z}{{\bf Z}}

% this is for latex 2.09
% unfortunately, the result is slanted - use Latex2e instead
%\newcommand{\bfLambda}{\mbox{\boldmath$\Lambda$}}
% this is for Latex2e
\newcommand{\bfLambda}{\boldsymbol{\Lambda}}

% Yuan Qi's boldsymbol
\newcommand{\bsigma}{\boldsymbol{\sigma}}
\newcommand{\balpha}{\boldsymbol{\alpha}}
\newcommand{\bpsi}{\boldsymbol{\psi}}
\newcommand{\bphi}{\boldsymbol{\phi}}
\newcommand{\boldeta}{\boldsymbol{\eta}}
\newcommand{\Beta}{\boldsymbol{\eta}}
\newcommand{\btau}{\boldsymbol{\tau}}
\newcommand{\bvarphi}{\boldsymbol{\varphi}}
\newcommand{\bzeta}{\boldsymbol{\zeta}}

\newcommand{\blambda}{\boldsymbol{\lambda}}
\newcommand{\bLambda}{\mathbf{\Lambda}}
\newcommand{\bOmega}{\mathbf{\Omega}}
\newcommand{\bomega}{\mathbf{\omega}}
\newcommand{\bPi}{\mathbf{\Pi}}

\newcommand{\btheta}{\boldsymbol{\theta}}
\newcommand{\bpi}{\boldsymbol{\pi}}
\newcommand{\bxi}{\boldsymbol{\xi}}
\newcommand{\bSigma}{\boldsymbol{\Sigma}}

\newcommand{\bgamma}{\boldsymbol{\gamma}}
\newcommand{\bGamma}{\mathbf{\Gamma}}

\newcommand{\bmu}{\boldsymbol{\mu}}
\newcommand{\1}{{\bf 1}}
\newcommand{\0}{{\bf 0}}

% \newcommand{\comment}[1]{}

\newcommand{\bs}{\backslash}
\newcommand{\ben}{\begin{enumerate}}
\newcommand{\een}{\end{enumerate}}

 \newcommand{\notS}{{\backslash S}}
 \newcommand{\nots}{{\backslash s}}
 \newcommand{\noti}{{\backslash i}}
 \newcommand{\notj}{{\backslash j}}
 \newcommand{\nott}{\backslash t}
 \newcommand{\notone}{{\backslash 1}}
 \newcommand{\nottp}{\backslash t+1}
% \newcommand{\notz}{\backslash z}

\newcommand{\notk}{{^{\backslash k}}}
%\newcommand{\noti}{{^{\backslash i}}}
\newcommand{\notij}{{^{\backslash i,j}}}
\newcommand{\notg}{{^{\backslash g}}}
\newcommand{\wnoti}{{_{\w}^{\backslash i}}}
\newcommand{\wnotg}{{_{\w}^{\backslash g}}}
\newcommand{\vnotij}{{_{\v}^{\backslash i,j}}}
\newcommand{\vnotg}{{_{\v}^{\backslash g}}}
\newcommand{\half}{\frac{1}{2}}
\newcommand{\msgb}{m_{t \leftarrow t+1}}
\newcommand{\msgf}{m_{t \rightarrow t+1}}
\newcommand{\msgfp}{m_{t-1 \rightarrow t}}

\newcommand{\proj}[1]{{\rm proj}\negmedspace\left[#1\right]}
\newcommand{\argmin}{\operatornamewithlimits{argmin}}
\newcommand{\argmax}{\operatornamewithlimits{argmax}}

\newcommand{\dif}{\mathrm{d}}
\newcommand{\abs}[1]{\lvert#1\rvert}
\newcommand{\norm}[1]{\lVert#1\rVert}

%miscellaneous symbols
\newcommand{\ie}{{{i.e.,}}\xspace}
\newcommand{\eg}{{{\em e.g.,}}\xspace}
\newcommand{\EE}{\mathbb{E}}
\newcommand{\VV}{\mathbb{V}}
\newcommand{\sbr}[1]{\left[#1\right]}
\newcommand{\rbr}[1]{\left(#1\right)}
\newcommand{\cmt}[1]{}



\footnotesize
	\begin{itemize}
		\item You are welcome to talk to other members of the class about
		the homework. I am more concerned that you understand the
		underlying concepts. However, you should write down your own
		solution. Please keep the class collaboration policy in mind.
		
		\item Feel free discuss the homework with the instructor or the TAs.
		
		\item Your written solutions should be brief and clear. You need to
		show your work, not just the final answer, but you do \emph{not}
		need to write it in gory detail. Your assignment should be {\bf no
			more than 10 pages}. Every extra page will cost a point.
		
		\item Handwritten solutions will not be accepted.
		
		\item The homework is due by \textbf{midnight of the due date}. Please submit
		the homework on Canvas.
	\end{itemize}



\section*{Warm up[100 points + 5 bonus]}	
\label{sec:q1}

%independency, conditional distribution, expectation, variance, basic properties
%gradient calcualtion, logistic function, second derivatives
%
\begin{enumerate}
%\item~[5 points] We use sets to represent events. For example, toss a fair coin $5$ times, and the event can be represented by the set of ``Heads" or ``Tails" after each tossing. Let a specific event $A$ be ``at least one head after tossing $5$ times". Calculate the probability that event $A$ happens, i.e., $p(A)$.


\item~[2 points] Given two events $A$ and $B$, prove that 
\begin{align}
&p(A \cup B) \le p(A) + p(B) \nonumber \\
&p(A \cap B) \le p(A), p(A \cap B) \le p(B) \nonumber
\end{align}
When does the equality hold?
\item~[2 points] Let $\{A_1, \ldots, A_n\}$ be a collection of events. Show that
\[
p(\cup_{i=1}^n A_i) \le \sum_{i=1}^n p(A_i).
\]
When does the equality hold? (Hint: induction)
%\item~[5 points] Given three events $A$, $B$ and $C$, show that
%\[
%p(A\cap B\cap C) = p(A|B\cap C)p(B|C)p(C)
%\]
\item~[14 points]  We use $\EE(\cdot)$ and $\VV(\cdot)$ to denote a random variable's mean (or expectation) and variance, respectively. Given two discrete random variables $X$ and $Y$, where $X \in \{0, 1\}$ and $Y \in \{0,1\}$. The joint probability $p(X,Y)$ is given in as follows:
\begin{table}[h]
        \centering
        \begin{tabular}{ccc}
        \hline\hline
         & $Y=0$ & $Y=1$ \\ \hline
         $X=0$ & $3/10$ & $1/10$ \\ \hline
         $X=1$  & $2/10$ & $4/10$ \\ \hline\hline
        \end{tabular}
        %\caption{Training data for the alien invasion problem.}\label{tb-alien-train}
        \end{table}
	
        \begin{enumerate}
            \item~[10 points] Calculate the following distributions and statistics. 
            \begin{enumerate}
            \item the the marginal distributions $p(X)$ and $p(Y)$
            \item the conditional distributions $p(X|Y)$ and $p(Y|X)$
            \item $\EE(X)$, $\EE(Y)$, $\VV(X)$, $\VV(Y)$
            \item  $\EE(Y|X=0)$, $\EE(Y|X=1)$,  $\VV(Y|X=0)$, $\VV(Y|X=1)$ 
            \item  the covariance between $X$ and $Y$
            \end{enumerate}
            \item~[2 points] Are $X$ and $Y$ independent? Why?
            \item~[2 points] When $X$ is not assigned a specific value, are $\EE(Y|X)$ and $\VV(Y|X)$ still constant? Why?
        \end{enumerate}
\item~[9 points] Assume a random variable $X$ follows a standard normal distribution, \ie $X \sim \N(X|0, 1)$. Let $Y = e^{-X^2}$. Calculate the mean and variance of $Y$.
\begin{enumerate}
	\item $\EE(Y)$
	\item $\VV(Y)$
	\item $\text{cov}(X, Y)$
\end{enumerate}

\item~[8 points] Derive the probability density functions of the following transformed random variables. 
\begin{enumerate}
	\item $X \sim \N(X|0, 1)$ and $Y = X^3$.
	%\item $\x =[x_1, x_2]^\top \sim \N(\x|\0, [1, -1/2;1/2,1])$ and $y =3 x_1 + 2x_2$. 
	\item $\left[\begin{array}{c}X_1 \\ X_2\end{array}\right] \sim \N\big(\left[\begin{array}{c}X_1 \\ X_2\end{array}\right]|\left[\begin{array}{c}0 \\ 0 \end{array}\right], \left[\begin{array}{cc}1 & -1/2 \\ -1/2 & 1\end{array}\right]\big)$ and $\left[\begin{array}{c}Y_1 \\ Y_2\end{array}\right] = \left[\begin{array}{cc}1 & 1/2\\-1/3&1\end{array}\right]\left[\begin{array}{c}X_1 \\ X_2\end{array}\right]$.
\end{enumerate}

\item~[10 points]  Given two random variables $X$ and $Y$, show that 
\begin{enumerate}
\item $\EE(\EE(Y|X)) = \EE(Y)$
\item
$\VV(Y) = \EE(\VV(Y|X)) + \VV(\EE(Y|X))$
\end{enumerate}
(Hints: using definition.)

%\item~[20 points]  Let us go back to the coin tossing example. Suppose we toss a coin for $n$ times, \textit{independently}. Each toss we have $\frac{1}{2}$ chance to obtain the head. Let us denote the total number of heads by $c(n)$. Derive the following statistics. You don't need to give the numerical values. You only need to provide the formula.
%\begin{enumerate}
%\item $\EE(c(1))$, $\VV(c(1))$
%\item $\EE(c(10))$, $\VV(c(10))$
%\item $\EE(c(n))$, $\VV(c(n))$
%\end{enumerate} 
%What can you conclude from comparing the expectations and variances with different choices of $n$?  

\item~[9 points] Given a logistic function, $f(\x) = 1/(1+\exp(-\a^\top \x))$ ($\x$ is a vector),  
\begin{enumerate}
\item derive $\frac{\d f(\x)}{\d \x}$
\item derive $\frac{\d^2 f(\x)}{\d \x^2}$, \ie the Hessian matrix
\item show that $-f(\x)$ is convex
\end{enumerate}
Note that $0 \le f(\x) \le 1$.

\item ~[10 points] Derive the convex conjugate for the following functions
\begin{enumerate}
\item $f(x) = -\log(x)$
\item $f(\x) = \x^\top \A^{-1} \x$ where $\A \succ 0$
\end{enumerate}

\item~[20 points] Derive the (partial) gradient of the following functions. Note that bold small letters represent vectors, bold captial letters matrices, and non-bold letters just scalars.
\begin{enumerate}
	\item $f(\x) = \x^\top \A \x$, derive $\frac{\partial f}{\partial \x}$
	\item $f(\x)  =\left(\I + \x\x^\top\right)^{-1} \x$, derive $\frac{\partial f}{\partial \alpha}$
	\item $f(\x) = \log |\K + \alpha \I|$, where $|\cdot|$ means the determinant. Derive $\frac{\partial f}{\partial \sigma}$
	\item $f(\bmu, \bSigma) = \log\big(\N(\a|\A\bmu, \S\bSigma\S^\top)\big)$, derive $\frac{\partial f}{\partial \bmu}$ and $\frac{\partial f}{\partial \bSigma}$,
	\item $f(\bSigma) = \log\big(\N(\a|\b, \K\otimes \bSigma)\big)$ where $\otimes$ is the Kronecker product (Hint: check Minka's notes).
\end{enumerate}
\item~[2 points] Given the multivariate Gaussian probability density, $$p(\x|\bmu, \bSigma) = |2\pi \bSigma|^{-\frac{1}{2}}\exp\left(- (\x-\bmu)^\top\bSigma^{-1}(\x-\bmu)\right).$$ Show that the density function achieves the maximum when $\x = \bmu$.  
\item~[5 points] Show that $$\int \exp(-\frac{1}{2\sigma^2}x^2) \d x = \sqrt{2\pi \sigma^2}.$$ Note that this is about  how the normalization constant of the Gaussian density is obtained. Hint: consider its square and use double integral. 
\item~[5 points] The gamma function is defined as $$\Gamma(x) = \int_0^\infty u^{x-1}e^{-u} \d u.$$ Show that $\Gamma(1) = 1$ and $\Gamma(x+1) = x\Gamma(x)$. Hint: using integral by parts. 
\item~[2 points] By using Jensen's inequality with $f(x) = \log(x)$, show that for any collection of positive numbers $\{x_1, \ldots, x_N\}$,
$$\frac{1}{N}\sum_{n=1}^N x_n \ge \left(\prod_{n=1}^N x_n\right)^{\frac{1}{N}}.$$
\item~[2 points] Given two probability density functions $p(\x)$ and $q(\x)$, show that $$\int p(\x)\log\frac{p(\x)}{q(\x)}\d \x \ge 0.$$
\item~[\textbf{Bonus}][5 points] Show that for any square matrix $\X \succ 0$, $\log\det\X$ is concave to $\X$. 
\end{enumerate}


\end{document}
%%% Local Variables:
%%% mode: latex
%%% TeX-master: t
%%% End:
